\chapter{Recommendations}
During the construction of the models, there are some parts that fall beyond this research. For future researches the following recommendations are made. Firstly, the model implementation in operational settings in Section \ref{model implementation in operational settings}. Secondly, the optimizing of the refit intervals in Section \ref{optimizing refit intervals}. For future researches conducted on the subject, some directions are mad in Section \ref{future research directions}. 

\section{Model Implementation in Operational Settings}
\label{model implementation in operational settings}
Implementing the currently designed advanced forecasting techniques in real-world situations should be done based on what needs to be forecasted. The Hybrid model is most applicable in routine scheduling, by generating 1-day forecasts. The model is relatively easy to implement and the improved accuracy, with the use of basic methods help to reduce operational uncertainty. The Bayesian Neural Network with Monte Carlo dropout is most applicable for high-risk operations, where the model can generate a predictive distribution of the forecasts. This can guide the operators to make a decision on when the operation could be executed. LSTM models can be used in regions that suffer from rapid weather changes, although the computational resources to refit often limits this forecast model.

\section{Optimizing Refit Intervals}
\label{optimizing refit intervals}
To be able to optimize the refit intervals it is important to know how the weather behaves of a certain time frame. Tailored intervals could give a solution, by using an adaptive refit interval which can be switched to a shorter interval during storms, the forecasts could be made more accurate. Keeping in mind this might not be beneficial for sites where a longer forecast is necessary.\\

\section{Future Research Directions}
\label{future research directions}
Where right now the research was focused on one site, a \textbf{Multi-Site Validation} would extend the analyses of these models. Different met-ocean conditions on different sites, with different depths could generalize the model performance and identify site-specific best practices. \textbf{Expanding the Parameter Set}, the models are now only based on their own trained values, exogenous parameters like wind direction, tides or swells could be included. This will probably lead to a model with higher accuracy, but at a cost of again a longer computational time. When models are implemented pm the sites \textbf{Real-Time Continuous Learning} could be applied. Finding a way to continuously retrain the model with real-time data, where the models will update/refit every time a new measurement is done. This is especially useful in short-term forecasting. \textbf{A Hybrid Probabilistic model} investigate if a combination of the Hybrid ARIMA-ANN model with the Bayesian Neural Network is possible. In this case the model would be able to not only make a accurate forecast, but also the distribution of the forecasts could be used in risk assessments. 