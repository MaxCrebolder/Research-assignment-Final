\chapter*{Abstract}
\setheader{Abstract}

An efficient and smooth installation of offshore wind-turbines, is highly influenced by the weather conditions. Accurate forecasting of the key weather parameter, significant wave height, mean wave frequency and  wind speed, is needed to minimize delays. These delays occur when operational limits of the vessels are reached and so downtime occurs, eventually leading to increased project costs. This research compares three advanced forecasting techniques: A Hybrid ARIMA-ANN model, which combines linear and non-linear components, a Bayesian Neural Network with Mont Carlo dropout, to find a probabilistic distribution in the solutions, a Long Short-Term Memory model, for long-term dependencies. All models where trained with 8 years, 2001-2008, of hourly data and tested on 2 years of data, 2009-2010, from a North Sea site. During the testing phase the models where refitted and retrained every 5 days. The Hybrid ARIMA-ANN model was further investigated and the effect of different refit intervals is investigated with a range from 6 hours to 4 weeks.\\

\noindent The models are evaluated with the following evaluation metrics, Mean Squared Error, Mean Absolute Error, Root Mean Squared Error and $R^2$, these reveal that frequent refit intervals load to an increase in accuracy but demands higher computational capability. The Hybrid ARIMA-ANN is the best performing model of the three, where it is able to capture the fluctuations in the data and in the mean time keeps the metrics to a minimum. In operational terms selecting an appropriate model will be site and operation specific, but the Hybrid model with a short refit interval \leq 1-day can reduce downtime and eventually lead to a smoother operation. The models where constructed in MatLab and can be found in GitHub 
